\documentclass[rmp, eprint, superscriptaddress,floatfix]{revtex4-1}
\usepackage{longtable}
\usepackage{color}
\usepackage{graphicx} % needed for figures
\usepackage[utf8x]{inputenc}
\usepackage{hyperref}
\usepackage{longtable}
\usepackage{xcolor}
\usepackage{soul}
\usepackage{multirow}
\usepackage{longtable}

\definecolor{drab}{rgb}{0.59, 0.44, 0.09}
\newcommand{\Richard}[1]{{\color{drab}Richard: #1}}
\definecolor{celestialblue}{rgb}{0.29, 0.59, 0.82}
\newcommand{\Robert}[1]{{\color{celestialblue}Robert: #1}}
\definecolor{purple}{rgb}{0.459,0.109,0.538}
\definecolor{deepsaffron}{rgb}{1.0, 0.6, 0.2}
\newcommand{\Jan}[1]{{\color{deepsaffron}Jan: #1}}
\newcommand{\Emma}[1]{{\color{purple}Emma: #1}}
\definecolor{green}{rgb}{0.15, 0.6, 0.15}
\newcommand{\Valentin}[1]{{\color{green}Valentin: #1}}


\begin{document}

Dear Editors of Eurosurveillance,

Please receive for your consideration our manuscript “Potential impact of seasonal forcing on a SARS-CoV-2 pandemic.” 

The novel virus SARS-CoV-2 has spread at an alarming pace in Wuhan and Hubei province, with exported cases and small transmission clusters reported globally. The WHO declared SARS-CoV-2 to be a ``\textit{public health emergency of international concern}" (Jan 30) and ``\textit{very grave threat for the rest of the world}” (Feb 11). There is much uncertainty on how the epidemic will unfold and what preventive measure to undertake. In this manuscript, we show that a decrease in case number incidence during the coming months might not only be an effect of preventive measures, but also seasonal variation of the transmissibility.

Drawing on ten years data on the circulation of the “common” coronaviruses (229E, OC43, NL63, HKU1) in Stockholm, Sweden, we model the effect of possible seasonal forcing on SARS-CoV-2. Since many parameter settings are unknown, we perform simulations for a range of plausible scenarios. In these simulations, we find that in the northern hemisphere a peak might occur either in the first half of 2020, in winter 2020/2021, or as two peaks with substantial regional variation.

As case counts will be closely monitored in the coming weeks and months, the scenarios we explore show that transient reductions in the incidence rate do not necessarily mean the pandemic is contained as they might be due to a combination of seasonal variation and infection control efforts. Furthermore, the simulations show that dynamics can differ greatly across sub-populations, meaning that case count trajectories in one country should be used cautiously to inform projections in a second country, even in the same climate. 

We think that these results should be taken into account in the further monitoring of the epidemic, and that they contain important information and implications to the readership of Eurosurveillance. 

We hope that you will consider our manuscript for publication through the rapid communication track and look forward to
your correspondence. We use the new name SARS-CoV-2 as suggested by ICTV, but would be happy to change to COVID-19 if needed.

Sincerely,

Richard A. Neher, on behalf of all authors


\end{document}